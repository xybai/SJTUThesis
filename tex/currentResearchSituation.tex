\chapter{国内外研究现状}

\section{国外主流在线数据交易平台}

Microsoft Windows Azure Data Marketplace\cite{MicrosoftAzure}在2010年正式成立,微软将其作为Azure云平台的一部分。Azure Data Marketplace非常善于将数据买家与数据发布者联系在一起。因此,几个主要的国外数据提供商比如ESRI, Dun and Bradstreet 都能在这个平台被找到。在Azure Data Marketplace上的所有数据集大致可以分为两类:免费和收费。这两种数据集都是需要消费者订阅才能有权限查阅或使用这些数据集的。对于免费的数据集,数据买家可以每月无限次数的访问它们,但在它们上定义的查询事务数量是有限的。而对于收费的数据集来说,数据买家需要根据每月访问次数来支付一定费用才能访问它们。通过使用一个标准的数据协议Odata,这个平台给数据买家提供了一个定义良好的网络接口来访问平台中的数据。然而,对于Azure Data Marketplace上的数据发布者,微软并没给出具体的定价建议,尤其是多个数据发布者存在竞争的局面时。

Infochimps\cite{infochimps}数据交易平台成立于2009年,它最初的目标是去收集尽可能多的公开的商业数据集。Factual\cite{factual}成立于2007年,它主要致力于在地理信息数据的收集和售卖。与Azure Data Marketplace存在同样的问题是,这二者都没有为平台上的数据发布商提供一个明确的定价策略建议。在Infochimps和Factual上,数据买家在购买自己想要的数据时只能去直接联系平台上的数据发布商,经协商后确定数据价格。这么看来,数据交易平台只是起到了一个中介的作用,最终的数据交易价格并不是统一和透明的。在表\ref{tab:comparison_datamarket}中,我们给出上述三个交易平台的详细比较。

\begin{table}[h]
\centering
\bicaption[tab:comparison_datamarket]{三个国外主要数据市场的比较}{三个国外主要数据市场的比较}{Table}{Comparison for three major data markets}
 \begin{tabular}{|p{2.5cm}|p{2.5cm}|p{2.5cm}|p{2.5cm}|}
        \hline
               & Azure~\cite{MicrosoftAzure} & Infochimps~\cite{infochimps} & Factual~\cite{factual} \\
        \hline
        \hline
          数据类型  & 多种类型 & 主要是地理数据 & 主要是地理、社交以及网络数据 \\
        \hline
          数据免费  & 是 & 是 & ---\\
        \hline
          付费数据的免费试用  & 是 & --- & 是,但仅供API免费试用 \\
        \hline
          交付方式  & OData API & API, 下载 & API, 为重度用户提供下载\\
        \hline
          应用部署  & Windows Azure & Infochimps自建平台 & --- \\
        \hline
          数据发布  & 通过网络服务或者连接数据库 & 上传 & 上传 \\
        \hline
        成立时间  & 2010 & 2009 & 2007 \\
        \hline
 \end{tabular}
\end{table}

\section{国内主流在线数据交易平台}

贵阳大数据交易所\cite{gbdex}成立于2014年12月31日,2015年4月14日正式挂牌运营,是我国乃至全球第一家大数据交易所。秉承“贡献中国数据智慧 释放全球数据价值”发展理念,志在成为全球最重要的交易所,旨在推动政府数据公开、行业数据价值发现。截至2016年9月1日,交易额累积突破1亿元,交易框架协议接近3亿元,发展会员超过500家,可交易数据产品接近4000个,可交易的数据总量超过60PB。贵阳大数据交易所交易的并不是底层数据,而是基于底层数据,通过数据的清洗、分析、建模、可视化出来的结果, 彻底解决了数据如何保护隐私及数据所有权的问题。 贵阳大数据交易所将成为永不休市的交易所, 将实行7×24小时的交易时间。其涉及的数据类型有金融、政府、医疗、社会、海关、能源、社交、商品、水电煤、法院、交通、企业、通信、银行卡、专利等。

武汉长江大数据交易中心是在武汉市委市政府支持下设立的、第三方中立的、具有公信力的大数据交易中心,是武汉市政府推出的“互联网+”产业创新工程“11711”行动计划中关于大数据产业发展的重要部署。武汉长江大数据交易中心采用市场化的运作方式,以推动政府及社会各领域数据的开放、融合为宗旨,以大数据应用为导向,汇聚数据清洗、数据加工、数据咨询、数据创意等全产业链资源,沉淀数据分析技术和供需场景,将多维度数据源与业务逻辑无缝衔接,逐步构建大数据交易生态,解决数据流通困局,让大数据真正成为推动区域经济转型升级的强大动力。

武汉东湖大数据交易中心\cite{chinadatatrading}成立于2015年7月,是经武汉市政府批准成立的华中地区首家大数据交易机构,也是国内最早探索并实施"政务数据运营解决方案"的服务机构,注册资金6000万元。东湖大数据联合武汉市互联网信息办公室和武汉市国有资产管理公司制订《武汉市政务数据资产运营中心成立方案》,参与筹建"武汉市政务数据资产运营中心",将成为全国首个以政务行业为主的大数据运营机构,将参与相关标准、规则的制定,来促进政府数据开放,带动整个产业链的发展。交易中心由具有政府背景的——武汉国有资产经营公司;领先的数据资产运营商——中润普达;优秀的地理空间信息公司——武大吉奥,及武汉市智慧产业投资公司、汉口银行等行业领军企业共同组建。交易中心目前整合的大数据覆盖了200多个行业、30大品类,实现了数千万条数据的汇集。截止2016年5月,已经服务了近百个省市区政府、金融机构、产业集团等客户。涉及到的数据集类型有交通环境、公共服务、健康医疗、金融商贸、科研应用、社交征信、科研应用、文娱音乐、知识产权、智慧生活、产业数据、政府数据。


\section{已有的数据定价策略}
\label{sec:data pricing strategy}
基于订阅的定价机制是一个传统的数据商品定价机制。在那些实行订阅定价机制的数据交易平台,数据买家需要根据预先设定的事务数支付访问数据的费用。Azure Data Market就是一个很好的例子来解释这种定价机制。Azure有两种按月订阅类型:有限型和无限型。表\ref{tab:azure_tariff}给出了Azure Data Market的一个具体的价目表。比如数据集2010 Key US Demographics就是完全受限类型。如果一个数据买家想要每个月访问这个数据集10次,那么他就要支付$\$$9.95。然而,数据集EU Health Data Service UK和Business Verification就与2010 Key US Demographics稍有不同了,因为他们是部分受限类型的。数据买家可以每个月以免费的价格访问EU Health Data Service UK 5次或者Business Verification 10次。一旦数据买家访问这些数据集超过相应的免费访问次数,那么他也会被收取相应的费用。虽然,数据拥有者能比较容易地采用基于订阅制的定价机制来给商品定价,但是如果订阅价格设计的不够精密时,就会出现套利现象,从而给数据卖家带来经济损失,同时也会对其他没有套利的买家造成不公平。

\begin{table}[h]
\centering
\bicaption[tab:azure_tariff]{Azure数据市场部分商品价目表}{Azure数据市场部分商品价目表}{Table}{Datasets tariff in Azure data market}

    \begin{tabular}{|c|c|l|}
     \hline
               Dataset~\cite{MicrosoftAzure}& Transaction Limit& Price Level \\
     \hline
     \hline
        \multirow{3}{2cm}{2010 Key US Demographics}

                & $10$ &$\$9.95$\\
                & $50$ &$\$24.95$\\
                & $150$ &$\$49.95$\\
         \hline
         \multirow{3}{2cm}{EU Health Data Service UK}

                & $5$ &$\$0.00$\\
                & $100$ &$\$293.09$\\
                & $200$ &$\$732.83$\\
         \hline
         \multirow{4}{2cm}{Business Verification}
                & $100$ &$\$0.00$\\
                & $200$ &$\$100.00$\\
                & $2,500$ &$\$1,250.00$\\
                & $5,000$ &$\$2,425.00$\\
        \hline
 \end{tabular}
\end{table}


基于查询的定价机制是来源于关系数据库中的query。最近,一些数据交易平台开始采用这种机制来售卖他们的数据集。具体来说,数据买家为其想要的数据集向数据平台发起不同的特定的请求,而数据平台返回相应数据集的视图作为查询结果给数据买家,数据卖家根据查询的复杂度收取一定的费用。比如,CustomList\cite{customlists}以$\$$399售卖其全美商业数据库。数据买家可能只在意该数据集中有邮件地址的公司的数据子集,那么买家就向交易平台发起这么一个查询,交易平台查到相应视图返回结果并向买家收取$\$$299。然而,现在的数据交易平台并不支持复杂的查询操作,因为目前仍然并不清楚如何给不同的查询结果定一个合适的价格。Koutris\cite{koutris2015query}等人提出了一个基于查询的数据定价框架,该框架允许卖家给一些基本视图事先赋一些价格,然后当买家发起查询时,将查询到的基本视图的价格的和作为查询费用。但是,他们的工作仍旧没有解决如何给数据集的基本视图赋予合适价格的问题。

捆绑销售定价机制是起源于资本数据市场,它代表了一种聚合技术\cite{bakos2001aggregation}。在资本数据市场中,卖家经常将其多种产品捆绑,并对不同的客户以不同的价格销售。因此,这就会产生价格歧视效应\cite{bakos1999bundling}。举例来说,Dow Jones是一个金融信息公司,它将其信息和其他在线服务(比如新闻邮件推送服务,对特定公司的新闻实行监控和过滤服务)。Dow Jones为订阅者免费提供部分信息服务,但如果订阅者想要检索信息的全部内容,就需要开始付费了。此外,现在一些信息公司也开始实施了根据信息内容深度进行区别定价的策略。比如,Dow Jones对新闻的标题收费$\$$0.20,对新闻摘要收费$\$$1.00,对完整的新闻收费$\$$3.50。需要指出的是只有当捆绑的产品具有负相关性时,捆绑销售策略才能被市场接受。对于文本信息商品,比如新闻、文章,人们能比较弄清地分辨标题、摘要和全文的关系,然后根据内容深度的不同来给它们定价。但是,现如今,大部分的信息商品是非结构化的数据,比如音频、图像和视频。数据拥有者想要辨清这些数值数据的内在关系是困难的,因此想要根据信息深度来定价是不太可行的。

 \section{商品在线拍卖}

拍卖一个古老但有效的定价机制,它最早出现在公元前500年。大量的商品通过拍卖这一形式被交易。William Vickrey是第一个提出系统的拍卖理论的研究者\cite{Vickrey1961Counterspeculation}。之后,在Vickrey的理论基础上,涌现了更多拍卖的研究工作\cite{Milgrom1989Auctions,Milgrom1982A,Riley1981Optimal}。Riley等人\cite{Riley1981Optimal}在拍卖者独立同分布假设下研究了最优拍卖的性质。具体来说,他们比较了不同拍卖机制下卖家的期望收益。他们发现对于多数拍卖规则,如果卖家不接受低于保留价的报价时,英式拍卖或者荷兰式拍卖能最大化卖家期望收益。

互联网的快速发展使得传统商品的在线拍卖变成了可能,最早的在线拍卖出现在1993年\cite{Lucking2000Auctions}。目前出现了一大批在线拍卖市场,比如国外的eBay Live Auction\footnote{eBay Live Auction官方网站: \url{https://www.ebay.com/rpp/live-auctions}},国内的淘宝拍卖会\footnote{淘宝拍卖会官方网站: \url{https://paimai.taobao.com}}。大多数关于在线拍卖的理论研究都集中于B2C (Business-to-Customer) 或者C2C (Customer-to-Customer)的。这很大原因是研究者本身充当的角色多数是Customer。Beam等人\cite{Beam1998Auctions}使用搜索引擎分析了100个B2C和C2C在线拍卖实例,Riley等人\cite{Riley1981Optimal}也通过类似的方式分析了142个B2C和C2C拍卖实例。他们两组人的研究得到了一些相似的结论,即目前在线数据拍卖只使用了传统且有限的四组拍卖机制:英式拍卖,荷兰式拍卖,第一价格拍卖,以及Vickrey拍卖。在文献\cite{Lucking2000Auctions}中,Lucking认为传统拍卖理论的假设是不适用于新兴出现的在线拍卖的。在线拍卖是非常不同于传统拍卖的。一些因素,比如拍卖持续时长、竞拍人数、拍卖起价和保留价等在传统拍卖和在线拍卖中起的作用不尽相同。Pinker等人\cite{Pinker2003Managing,Pinker2001Using}称延长在线拍卖时长能有助于抬高成交价。Roth等人\cite{Roth2002Last}研究发现延迟进入拍卖 (late bidding) 和狙击拍卖 (snipe bidding) 在在线拍卖场景中时常发生。Reiley等人\cite{Ockenfels2006Online}发现设置公开的起拍价和保留价会减少竞拍人数且会使得竞拍物品流拍的可能性增大。此外,Ariely\cite{Dan2003Buying}也发现起拍价和最终成交价有着正相关关系。

大多数传统商品都是通过标的价格进行售卖的,但越来越多的物品开始通过拍卖这种机制来进行交易。在实践中,交易参与者们越来越多地认识到这种交易机制的好处。尽管大多数消费者对标的价格交易机制很熟悉而且也默认这种机制为大多数物品的交易方式,但是从交易量上来说,这是一个错误的认知。如果物品的交易成本和复杂度越高,则该物品越有可能通过拍卖的方式进行交易\cite{Pinker2003Managing}。此外,拍卖成交价能真实反映拍卖胜者的支付意愿,这能最大化拍卖参与者的总剩余。Lu等人\cite{Lu1996The}总结了这些通过拍卖交易的商品的共同特性:1、唯一性;2、不确定均衡价格。数据商品恰好符合这两个特性。在目前的在线数据市场,买家的数量远远大于卖家数量,这意味着数据交易市场是一个不完全竞争市场,近似于寡头市场。Harris等人\cite{Harris1981A}发现当市场需求超过供应时,拍卖也许是最好的交易机制。大量事实表明拍卖也许是一个对数据商品定价能起到最佳指导的机制。近些年来,大量关于传统商品的在线拍卖的研究\cite{Pinker2003Managing,Lucking2000Auctions,Pinker2001Using,Ockenfels2006Online,Dan2003Buying},例如邮票、古董等。然后传统商品的在线拍卖的经验是不能直接移植到数据商品上。另一方面,因为缺乏足够多的对于数据商品的评价指标,造成数据买卖双方不能很好地对待售商品进行准确的估价。

\section{本章小结}

